% Options for packages loaded elsewhere
\PassOptionsToPackage{unicode}{hyperref}
\PassOptionsToPackage{hyphens}{url}
%
\documentclass[
]{article}
\usepackage{amsmath,amssymb}
\usepackage{lmodern}
\usepackage{iftex}
\ifPDFTeX
  \usepackage[T1]{fontenc}
  \usepackage[utf8]{inputenc}
  \usepackage{textcomp} % provide euro and other symbols
\else % if luatex or xetex
  \usepackage{unicode-math}
  \defaultfontfeatures{Scale=MatchLowercase}
  \defaultfontfeatures[\rmfamily]{Ligatures=TeX,Scale=1}
\fi
% Use upquote if available, for straight quotes in verbatim environments
\IfFileExists{upquote.sty}{\usepackage{upquote}}{}
\IfFileExists{microtype.sty}{% use microtype if available
  \usepackage[]{microtype}
  \UseMicrotypeSet[protrusion]{basicmath} % disable protrusion for tt fonts
}{}
\makeatletter
\@ifundefined{KOMAClassName}{% if non-KOMA class
  \IfFileExists{parskip.sty}{%
    \usepackage{parskip}
  }{% else
    \setlength{\parindent}{0pt}
    \setlength{\parskip}{6pt plus 2pt minus 1pt}}
}{% if KOMA class
  \KOMAoptions{parskip=half}}
\makeatother
\usepackage{xcolor}
\IfFileExists{xurl.sty}{\usepackage{xurl}}{} % add URL line breaks if available
\IfFileExists{bookmark.sty}{\usepackage{bookmark}}{\usepackage{hyperref}}
\hypersetup{
  pdftitle={The Causal Impact of Romneycare on Massachusetts' Mortality Rate using Synthetic Control},
  pdfauthor={Elle Boodsakorn, Parker Gauthier, Amal Kadri, Alice Kemp and Austin Longoria},
  hidelinks,
  pdfcreator={LaTeX via pandoc}}
\urlstyle{same} % disable monospaced font for URLs
\usepackage[margin=1in]{geometry}
\usepackage{graphicx}
\makeatletter
\def\maxwidth{\ifdim\Gin@nat@width>\linewidth\linewidth\else\Gin@nat@width\fi}
\def\maxheight{\ifdim\Gin@nat@height>\textheight\textheight\else\Gin@nat@height\fi}
\makeatother
% Scale images if necessary, so that they will not overflow the page
% margins by default, and it is still possible to overwrite the defaults
% using explicit options in \includegraphics[width, height, ...]{}
\setkeys{Gin}{width=\maxwidth,height=\maxheight,keepaspectratio}
% Set default figure placement to htbp
\makeatletter
\def\fps@figure{htbp}
\makeatother
\setlength{\emergencystretch}{3em} % prevent overfull lines
\providecommand{\tightlist}{%
  \setlength{\itemsep}{0pt}\setlength{\parskip}{0pt}}
\setcounter{secnumdepth}{-\maxdimen} % remove section numbering
\usepackage{booktabs}
\usepackage{longtable}
\usepackage{array}
\usepackage{multirow}
\usepackage{wrapfig}
\usepackage{float}
\usepackage{colortbl}
\usepackage{pdflscape}
\usepackage{tabu}
\usepackage{threeparttable}
\usepackage{threeparttablex}
\usepackage[normalem]{ulem}
\usepackage{makecell}
\usepackage{xcolor}
\ifLuaTeX
  \usepackage{selnolig}  % disable illegal ligatures
\fi

\title{The Causal Impact of Romneycare on Massachusetts' Mortality Rate
using Synthetic Control}
\author{Elle Boodsakorn, Parker Gauthier, Amal Kadri, Alice Kemp and
Austin Longoria}
\date{University of Texas at Austin - ECO395M - Spring 2022}

\begin{document}
\maketitle

\hypertarget{abstract}{%
\section{Abstract}\label{abstract}}

\newpage

\hypertarget{introduction}{%
\section{Introduction}\label{introduction}}

Access to healthcare through affordable health insurance remains one of
the most divisive policy debates in the United States, with critics of a
privatized system arguing in favor of a widespread, easily accessible
option targeted at low income households to increase the quality and
quantity of health care. However, some supporters of private insurance
argue against such a system, stating that insurance provided by the
federal government unfairly burdens taxpayers, many of whom opt for
private coverage through their employers. Before the groundbreaking
introduction of the Affordable Care Act in 2008, better known as
``Obamacare'', the state of Massachusetts passed its own state-provided
health insurance program signed into law by Senator Mitt Romney in 2006.
The program, dubbed ``Romneycare'' and the first of its kind in the
U.S., provided free and heavily subsidized health insurance to the
lowest income resident of the state, and mandated that nearly all
residents obtain a minimal level of coverage. To aid in the provision of
coverage to higher earners, the law also required every employer in the
state with over ten full-time employees to provide a health insurance
plan to workers. Before the program was heavily rolled back in favor of
Obamacare in 2012, over 97\% of Massachusetts residents had health
insurance coverage (\textbf{cite - brookings}). Many studies have
investigated the effects of Romneycare and Obamacare on statistics such
as health coverage, insurance utilization, and health care pricing,
however, few have attempted to isolate the causal effect of mandated
affordable health coverage on mortality rates. One of the reasons why
this effect can be difficult to identify is due to the distribution of
health care in the U.S. - as insurance is not randomly assigned to
individuals but normally obtained through employers, low income
individuals tend to have minimal coverage, if at all and sub-optimal
quality of care. As a result, worse health outcomes tend to be
correlated with individuals on the lower end of the socioeconomic
spectrum, making any causal analysis particularly complex. However,
using causal inference techniques such as Synthetic Control, we will be
able to isolate the direct effect of Romneycare on mortality outcomes.
Overall, we theorize that the nearly maximized population of residents
with mandated health insurance coverage required by Romneycare will
decrease mortality due to more individuals seeking both preventative and
emergency care without the burden of uninsured pricing.

\hypertarget{previous-literature}{%
\section{Previous Literature}\label{previous-literature}}

In this report, we will attempt to replicate Sommers, Long and Baicker's
2014 study of changes in Massachusetts' mortality after the introduction
of Romneycare and the Massachusetts Health Care Reform. In their
analysis, the authors utilized a natural experiment by comparing
population-level Massachusetts mortality obtained from the Center for
Disease Control (CDC) following the implementation of Romneycare to
counties covering approximately 25\% of the United States. The counties
included were selected to best match the racial makeup, gender balance,
age cohorts, and baseline death rates found in Massachusetts. Using a
propensity score framework, the authors use a regression-based
methodology to ``match'' the untreated counties to Massachusetts based
on pre-treatment characteristics and mortality rates.

The results, published in the American College of Physicians' Annals of
Internal Medicine, found an overall annual decrease of 320 deaths per
year in Massachusetts as compared to counties outside the state.

\hypertarget{data}{%
\section{Data}\label{data}}

To build the synthetic control, state-level annual time-series data was
extracted from the Census Bureau covering the period between 2000 and
2010. Relevant data collected included gender balance; racial diversity
split into White, Black, Asian, American Indian/Native, Hawaiian/Pacific
Islander, and multiracial as percent of population; age cohorts as
percent of population; unemployment rate; poverty rate; median household
income; and uninsured rate as the percent of residents without health
insurance coverage.

\hypertarget{methodology}{%
\section{Methodology}\label{methodology}}

Let \(Y_{it}\) be the mortality rate for state \(i\) at time period
\(t\) and let \(D_i \in (0,1)\) denote the binary treatment indicator of
state \(i\). Our goal is to estimate the average treatment effect of
introducing Romneycare on mortality rates in Massachusetts, or:
\[ATT_t = E[Y_{it}^1 - Y_{it}^0 | D_i = 1] = E[Y_{it}^1|D_i = 1] - E[Y_{it}^0|D_i = 1]\]
In words, the ATT can be calculated as the estimate of the difference in
outcomes for the treated units in the world where they were treated
versus the world where they were not treated. The key crux to this
calculation lies in our ability to measure \(Y_{it}^0|D_i=1\), or how
the treated units would have performed had they not been treated. This
is where the idea of Synthetic Control comes into play. The Synthetic
Control estimator interpolates this unknown by using a weighted average
of the untreated units to create a synthetic untreated unit with
pre-treatment characteristics similar to those of the treated unit. In
our analysis, we will use cross-sectional time-series mortality data
from counties outside of Massachusetts that did not have affordable care
laws that most closely match the demographic makeup of Massachusetts.
Using SC, an optimized weighted average of these selected counties'
mortality rates will serve as our untreated unit to compare to
Massachusetts' mortality before and after Romneycare was enacted. In
comparison to Sommers et al's approach which uses propensity score
matching, synthetic control uses optimization of individual weights,
which avoids the small sample bias that propensity score matching is
susceptible to when only one treatment unit is used, the state of
Massachusetts in our case.

\hypertarget{analysis}{%
\section{Analysis}\label{analysis}}

\textbackslash begin\{table\}{[}!h{]}

\caption{\label{tab:sc}Synthetic Control: Weights of sampled states}

\centering
\begin{tabu} to \linewidth {>{\raggedright}X>{\raggedleft}X}
\toprule
State & Weight\\
\midrule
Alabama & 0.00002\\
Alaska & 0.00000\\
Arizona & 0.00000\\
Arkansas & 0.00001\\
California & 0.00001\\
\addlinespace
Colorado & 0.00000\\
Connecticut & 0.10762\\
Delaware & 0.00001\\
District of Columbia & 0.08923\\
Florida & 0.00000\\
\addlinespace
Georgia & 0.00019\\
Hawaii & 0.00008\\
Idaho & 0.00000\\
Illinois & 0.00003\\
Indiana & 0.00002\\
\addlinespace
Iowa & 0.00001\\
Kansas & 0.00001\\
Kentucky & 0.00006\\
Louisiana & 0.00001\\
Maine & 0.00009\\
\addlinespace
Maryland & 0.33307\\
Michigan & 0.00007\\
Minnesota & 0.05727\\
Mississippi & 0.00002\\
Missouri & 0.00002\\
\bottomrule
\end{tabu}
\centering

\textbackslash begin\{tabu\}{[}t{]}\{llr\} \toprule \& State \&
Weight\textbackslash{} \midrule
26 \& Montana \& 0.00000\textbackslash{} 27 \& Nebraska \&
0.00001\textbackslash{} 28 \& Nevada \& 0.00000\textbackslash{} 29 \&
New Hampshire \& 0.00010\textbackslash{} 30 \& New Jersey \&
0.07143\textbackslash{} \addlinespace
31 \& New Mexico \& 0.00000\textbackslash{} 32 \& New York \&
0.00229\textbackslash{} 33 \& North Carolina \& 0.00002\textbackslash{}
34 \& North Dakota \& 0.00000\textbackslash{} 35 \& Ohio \&
0.00003\textbackslash{} \addlinespace
36 \& Oklahoma \& 0.00000\textbackslash{} 37 \& Oregon \&
0.00000\textbackslash{} 38 \& Pennsylvania \& 0.00000\textbackslash{} 39
\& Rhode Island \& 0.31573\textbackslash{} 40 \& South Carolina \&
0.00002\textbackslash{} \addlinespace
41 \& South Dakota \& 0.00000\textbackslash{} 42 \& Tennessee \&
0.00048\textbackslash{} 43 \& Texas \& 0.00000\textbackslash{} 44 \&
Utah \& 0.00000\textbackslash{} 45 \& Vermont \& 0.00002\textbackslash{}
\addlinespace
46 \& Virginia \& 0.02162\textbackslash{} 47 \& Washington \&
0.00001\textbackslash{} 48 \& West Virginia \& 0.00001\textbackslash{}
49 \& Wisconsin \& 0.00033\textbackslash{} 50 \& Wyoming \&
0.00000\textbackslash{} \bottomrule \textbackslash end\{tabular\}
\textbackslash end\{table\}

\hypertarget{conclusion}{%
\section{Conclusion}\label{conclusion}}

\end{document}
